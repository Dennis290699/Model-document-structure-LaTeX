\section{Principios y fundamentos de Kanban}

La metodología Kanban se fundamenta en un conjunto de principios y prácticas orientados a optimizar el flujo de trabajo, mejorar la eficiencia de los equipos y promover una cultura de mejora continua. A diferencia de otras metodologías ágiles que imponen estructuras estrictas, Kanban propone una evolución orgánica de los procesos existentes, adaptándose a las condiciones reales del entorno de trabajo.

\subsection{Principios básicos de Kanban}

Según David J. Anderson, los principios de Kanban se dividen en dos grupos: los principios de cambio y los principios de entrega de servicios. Ambos proporcionan un marco conceptual que guía la implementación progresiva de la metodología.

\subsubsection*{1. Principios de cambio}
\begin{itemize}
    \item \textbf{Comenzar con lo que se está haciendo actualmente:} Kanban no requiere reorganizar el proceso actual. Se inicia con los flujos, funciones y roles existentes.
    \item \textbf{Aceptar el cambio evolutivo:} En lugar de cambios bruscos, Kanban promueve una mejora continua, paso a paso.
    \item \textbf{Respetar los roles y responsabilidades actuales:} Se reconoce el valor de las estructuras organizativas existentes y se busca mejorarlas gradualmente.
\end{itemize}

\subsubsection*{2. Principios de entrega de servicios}
\begin{itemize}
    \item \textbf{Enfocarse en las necesidades del cliente:} Las decisiones se orientan hacia la generación de valor para el cliente.
    \item \textbf{Gestionar el trabajo, no a las personas:} El enfoque está en el flujo de tareas y no en el control del equipo.
    \item \textbf{Revisar regularmente el rendimiento del sistema:} Se promueve la mejora del sistema de entrega como un todo.
\end{itemize}

\subsection{Prácticas fundamentales de Kanban}

El éxito de Kanban radica en la aplicación rigurosa de sus prácticas fundamentales, las cuales han demostrado ser eficaces para visualizar, gestionar y optimizar el flujo de trabajo. Estas prácticas son interdependientes y conforman el núcleo operativo de la metodología.

\subsubsection{Visualización del trabajo}

La visualización del flujo de trabajo es el pilar esencial de Kanban. Mediante tableros (físicos o digitales), las tareas se representan como tarjetas dispuestas en columnas que reflejan los distintos estados del proceso (por ejemplo: ``Por hacer'', ``En proceso'', ``Hecho''). Esta representación permite a todos los miembros del equipo identificar cuellos de botella, prioridades y el estado general del sistema de un solo vistazo.

\subsubsection{Limitación del trabajo en curso (WIP)}

El \textit{Work In Progress} (WIP) o trabajo en curso se limita intencionalmente para evitar la saturación del sistema. Esta práctica busca que los equipos finalicen las tareas antes de iniciar nuevas, lo que reduce el tiempo de ciclo, mejora la calidad y evita el multitasking improductivo. El control del WIP también fomenta un flujo de trabajo más equilibrado y predecible.

\subsubsection{Gestión del flujo}

Kanban se centra en el flujo continuo de valor. La gestión del flujo implica monitorear y optimizar cómo se mueven las tareas a través del sistema, asegurando que no haya estancamientos o bloqueos prolongados. Métricas como el tiempo de ciclo (\textit{cycle time}) y el tiempo de entrega (\textit{lead time}) permiten evaluar la eficiencia del flujo y aplicar mejoras sistemáticas.

\subsubsection{Establecimiento de políticas explícitas}

Para que el sistema funcione de forma coherente, es fundamental establecer y comunicar reglas claras. Estas políticas pueden incluir criterios para mover tareas entre columnas, condiciones de aceptación o procedimientos ante bloqueos. Al hacer las políticas explícitas, se evita la ambigüedad y se mejora la autonomía del equipo.

\subsubsection{Circuitos de retroalimentación}

La retroalimentación regular es clave para la mejora continua. Kanban promueve la inclusión de reuniones periódicas como:
\begin{itemize}
    \item \textbf{Revisión de flujo:} análisis del rendimiento del sistema.
    \item \textbf{Reuniones de retrospectiva:} identificación de oportunidades de mejora.
    \item \textbf{Reuniones diarias (opcionales):} seguimiento del trabajo en curso.
\end{itemize}

Estos ciclos refuerzan la transparencia, la colaboración y el aprendizaje continuo dentro del equipo.

\subsubsection{Mejora continua (\textit{Kaizen})}

Inspirado en la filosofía japonesa \textit{Kaizen}, Kanban adopta una mentalidad de mejora constante. Los equipos observan su sistema de trabajo, identifican áreas de oportunidad y experimentan con ajustes incrementales. Esta práctica permite que el sistema evolucione orgánicamente en respuesta a cambios internos y externos.

\subsection{Síntesis de los fundamentos de Kanban}

Los principios y prácticas de Kanban no solo brindan una base sólida para la gestión de proyectos, sino que también fomentan una cultura de responsabilidad, transparencia y excelencia operativa. Su naturaleza adaptable y su enfoque en la mejora evolutiva lo convierten en una herramienta versátil, capaz de integrarse con éxito en una amplia variedad de contextos organizacionales.
