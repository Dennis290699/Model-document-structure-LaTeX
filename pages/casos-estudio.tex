\section{Casos de estudio o ejemplos prácticos}

La aplicación de la metodología Kanban ha trascendido el ámbito del desarrollo de software, extendiéndose con éxito a diversos sectores industriales y de servicios. En esta sección se presentan ejemplos prácticos que evidencian cómo la implementación de Kanban ha contribuido a mejorar la eficiencia operativa, optimizar flujos de trabajo y fomentar una cultura de mejora continua.

\subsection{Caso 1: Desarrollo de software en una startup tecnológica}

Una startup dedicada al desarrollo de aplicaciones móviles adoptó Kanban para gestionar su proceso de desarrollo. Inicialmente, el equipo enfrentaba problemas como sobrecarga de tareas, falta de visibilidad sobre el estado del proyecto y retrasos recurrentes en las entregas.

\begin{itemize}
    \item \textbf{Acciones implementadas:} se diseñó un tablero Kanban digital con columnas como \textit{To Do}, \textit{En Progreso}, \textit{En Pruebas} y \textit{Terminado}. Se establecieron límites WIP y se realizaron reuniones semanales de revisión de flujo.
    \item \textbf{Resultados:} en tres meses, se redujo el tiempo promedio de entrega de funcionalidades en un 40\%, se detectaron cuellos de botella en la etapa de pruebas y se mejoró la colaboración entre los desarrolladores y el área de control de calidad.
\end{itemize}

\subsection{Caso 2: Atención al cliente en una empresa de telecomunicaciones}

Una empresa nacional de telecomunicaciones incorporó Kanban en su centro de atención al cliente con el objetivo de mejorar la gestión de incidencias técnicas reportadas por usuarios.

\begin{itemize}
    \item \textbf{Acciones implementadas:} el equipo utilizó un tablero físico que representaba cada etapa del proceso de resolución de incidentes (Recepción, Diagnóstico, Solución, Cierre). Se asignaron responsables visibles en cada tarea y se midieron los tiempos de resolución promedio.
    \item \textbf{Resultados:} la tasa de resolución en primera llamada aumentó un 25\%, mientras que los tiempos de atención bajaron de 48 a 28 horas en promedio. Además, se logró una mayor transparencia en la carga de trabajo de cada agente.
\end{itemize}

\subsection{Caso 3: Gestión de marketing digital}

Un equipo de marketing de una agencia publicitaria decidió aplicar Kanban para gestionar sus campañas publicitarias en redes sociales. Antes de la implementación, los plazos de entrega eran poco predecibles y las tareas se acumulaban en la etapa de diseño gráfico.

\begin{itemize}
    \item \textbf{Acciones implementadas:} el tablero incluyó columnas específicas como \textit{Brief recibido}, \textit{Diseño en proceso}, \textit{Revisión del cliente}, \textit{Programado} y \textit{Publicado}. Se colocaron límites WIP y se priorizaron tareas urgentes con tarjetas de color.
    \item \textbf{Resultados:} se logró una planificación más precisa de publicaciones semanales, el equipo de diseño redujo el tiempo de respuesta en un 30\% y se minimizó la duplicación de esfuerzos.
\end{itemize}

\subsection{Análisis transversal de los casos}

Los tres casos anteriores demuestran que Kanban puede adaptarse a diversos contextos organizacionales, siempre que se respete el principio de visualizar el trabajo, gestionar el flujo y aplicar mejoras de forma incremental. Entre los beneficios observados se destacan:

\begin{itemize}
    \item Aumento de la eficiencia y visibilidad del proceso.
    \item Identificación de cuellos de botella.
    \item Mayor responsabilidad compartida y autonomía del equipo.
    \item Flexibilidad frente a cambios inesperados o tareas urgentes.
\end{itemize}

Estos resultados refuerzan la utilidad de Kanban como metodología versátil y de bajo costo de adopción para entornos que requieren adaptabilidad, colaboración y mejora continua.
