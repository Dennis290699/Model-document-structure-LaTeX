\section{Objetivos}

La presente investigación grupal tiene como propósito fundamental analizar y comprender de manera profunda la metodología ágil \textbf{Kanban}, una de las herramientas más representativas en la gestión eficiente del trabajo y la mejora continua dentro del ámbito del desarrollo de software y otros entornos productivos.

\vspace{0.5cm}
\noindent A continuación, se detallan los objetivos generales y específicos que guiaron el desarrollo de este informe técnico:

\subsection*{Objetivo general}
\begin{itemize}
    \item Investigar y documentar de forma detallada la metodología Kanban, abordando sus fundamentos, principios, fases operativas, ventajas, herramientas asociadas y su aplicabilidad en entornos de ingeniería de software.
\end{itemize}

\subsection*{Objetivos específicos}
\begin{itemize}
    \item Analizar los orígenes históricos y evolución del enfoque Kanban desde la industria manufacturera hasta su adaptación en la ingeniería de software.
    \item Describir los principios fundamentales que rigen el funcionamiento de Kanban y su enfoque centrado en la visualización del trabajo y la gestión del flujo.
    \item Identificar y caracterizar las fases del flujo de trabajo típico dentro de un tablero Kanban.
    \item Examinar las ventajas y desventajas que presenta esta metodología frente a otros marcos ágiles como Scrum o XP.
    \item Exponer casos de estudio reales que evidencien el impacto de Kanban en equipos de trabajo multidisciplinarios.
    \item Promover el análisis crítico y colaborativo entre los miembros del grupo como parte del proceso formativo dentro de la cátedra de Ingeniería de Software.
\end{itemize}
