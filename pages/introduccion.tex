\section{Introducción}

En el actual panorama de la ingeniería de software, donde la adaptabilidad, la eficiencia y la entrega continua de valor son requisitos esenciales, las metodologías ágiles han emergido como enfoques dominantes para la gestión de proyectos. Dentro de este ecosistema ágil, \textbf{Kanban} destaca por su enfoque visual, flexible y progresivo hacia la optimización del flujo de trabajo.

La presente investigación ha sido desarrollada de manera colaborativa por un grupo de estudiantes con el objetivo de ahondar en la comprensión estructural, operativa y práctica de Kanban. Esta metodología, originada en el sistema de producción de Toyota, ha trascendido su contexto industrial para convertirse en una herramienta de gran utilidad en el desarrollo de software, la atención al cliente, el marketing, la educación, entre otros sectores. Su capacidad para adaptarse sin necesidad de cambiar completamente la estructura existente de trabajo la hace especialmente atractiva para equipos que buscan mejorar sin interrupciones bruscas.

El informe que a continuación se presenta responde a la necesidad académica de adquirir conocimientos robustos sobre los marcos ágiles contemporáneos, en específico, sobre Kanban. A lo largo del documento, se exploran sus antecedentes, principios fundamentales, componentes clave, fases del flujo de trabajo, roles posibles, herramientas digitales de apoyo, ventajas comparativas, y se ilustran sus beneficios mediante ejemplos reales de implementación. 

Además, este trabajo no solo tiene un valor teórico, sino también formativo. Cada sección ha sido elaborada con una intención pedagógica clara, motivando el trabajo colaborativo, la investigación individual y el análisis crítico por parte del grupo, en concordancia con los objetivos de la cátedra de Ingeniería de Software.

Así, este estudio se convierte en una herramienta de doble propósito: aportar al conocimiento colectivo del grupo investigador y ofrecer a otros estudiantes una guía clara y detallada sobre Kanban y su aplicabilidad práctica.
