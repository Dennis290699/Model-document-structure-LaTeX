\section{Ventajas y desventajas de Kanban}

El enfoque Kanban, derivado de los principios Lean, ha demostrado ser una herramienta eficaz en la gestión de proyectos, especialmente en contextos donde la flexibilidad, la mejora continua y la visualización del trabajo son esenciales. No obstante, como toda metodología, presenta beneficios significativos, pero también limitaciones que deben ser consideradas al momento de su implementación.

\subsection{Ventajas de Kanban}

\begin{itemize}
    \item \textbf{Visualización clara del trabajo:} Uno de los mayores beneficios de Kanban es la representación gráfica del flujo de tareas, lo que permite al equipo identificar cuellos de botella, retrasos y sobrecarga de trabajo de forma inmediata.
    
    \item \textbf{Incremento en la eficiencia:} Al limitar el trabajo en curso (WIP) y promover el enfoque en tareas específicas, se reducen las interrupciones y se mejora el rendimiento general del equipo.
    
    \item \textbf{Adaptabilidad y flexibilidad:} A diferencia de metodologías como Scrum, Kanban no impone iteraciones fijas, por lo que es ideal para entornos con requerimientos cambiantes o impredecibles.
    
    \item \textbf{Mejora continua (Kaizen):} La retroalimentación constante, junto con la revisión de métricas del flujo, facilita procesos de mejora progresiva basados en datos reales y decisiones colaborativas.
    
    \item \textbf{Facilidad de adopción:} Kanban puede integrarse sin necesidad de reestructurar por completo la organización o redefinir roles, lo cual reduce la resistencia al cambio.
    
    \item \textbf{Mayor transparencia y alineación:} Al hacer visibles las políticas de trabajo, los estados de las tareas y los límites, se promueve una cultura de responsabilidad compartida y toma de decisiones consensuada.
\end{itemize}

\subsection{Desventajas de Kanban}

\begin{itemize}
    \item \textbf{Falta de estructura formal:} Al carecer de roles definidos y de ciclos de trabajo estrictos, Kanban puede resultar confuso para equipos que requieren dirección clara o estructuras más rígidas.
    
    \item \textbf{Dificultad para escalar:} Aunque se puede adaptar a contextos más amplios, Kanban no ofrece, de forma nativa, un marco completo para la gestión de múltiples equipos o grandes proyectos sin personalización adicional.
    
    \item \textbf{Riesgo de acumulación de tareas:} Si no se establecen y respetan correctamente los límites WIP, pueden generarse embudos de trabajo que comprometan el flujo y la calidad de las entregas.
    
    \item \textbf{Dependencia del compromiso del equipo:} El éxito de Kanban depende en gran medida de la disciplina y el compromiso del equipo para seguir las políticas, mantener actualizado el tablero y revisar el desempeño.
    
    \item \textbf{Menor visibilidad de largo plazo:} Al centrarse en el flujo continuo y no en entregas por iteraciones, puede dificultarse la planificación a mediano o largo plazo si no se combinan herramientas complementarias.
\end{itemize}

\subsection{Consideraciones finales}

La elección de Kanban como metodología de trabajo debe estar alineada con las necesidades, cultura y madurez del equipo. Sus beneficios se maximizan en contextos donde se valora la autonomía, la mejora progresiva y la adaptabilidad. No obstante, requiere un entorno colaborativo y comprometido para evitar caer en una gestión desorganizada o poco efectiva.

