\section*{Resumen}

Este informe académico presenta un estudio detallado y profundo sobre la metodología ágil \textbf{Kanban}, abordando su origen, evolución, fundamentos teóricos, componentes clave, fases del flujo de trabajo, así como su implementación práctica en diversos contextos. La investigación ha sido desarrollada de manera colaborativa por un equipo de estudiantes de Ingeniería de Software, con el objetivo de comprender y analizar el impacto de Kanban en la gestión eficiente de proyectos, particularmente en el desarrollo de software.

Kanban, cuyo origen se remonta al sistema de producción de Toyota bajo los principios del \textit{Lean Manufacturing}, ha sido adaptado exitosamente al ámbito de la ingeniería del software y la gestión de proyectos. Su enfoque basado en la visualización del trabajo, la limitación del trabajo en curso (WIP), la gestión continua del flujo y la mejora constante (Kaizen), ha permitido a las organizaciones alcanzar mayores niveles de eficiencia, flexibilidad y transparencia.

A lo largo del documento se exploran los principios fundamentales de Kanban, sus elementos estructurales (como los tableros, tarjetas, columnas y límites WIP), las fases típicas del flujo de trabajo y las buenas prácticas para su implementación. Además, se realiza una comparación analítica con otras metodologías ágiles como Scrum y XP, destacando las ventajas relativas de Kanban, especialmente en entornos donde se requiere adaptabilidad progresiva sin cambios disruptivos.

Complementariamente, se incluyen casos de estudio y ejemplos reales que ilustran cómo Kanban ha sido aplicado con éxito tanto en el sector tecnológico como en otros ámbitos organizacionales. Finalmente, se reflexiona sobre las lecciones aprendidas y la relevancia de Kanban como herramienta de transformación organizacional y como competencia esencial para los profesionales de la ingeniería de software del siglo XXI.

\vspace{0.5cm}
\noindent\textbf{Palabras clave:} Kanban, metodologías ágiles, gestión de proyectos, flujo de trabajo, mejora continua, Lean, software.
