\section*{Glosario de términos}

A continuación, se presenta un glosario con términos clave utilizados a lo largo del informe, con el fin de facilitar la comprensión del lector:

\begin{description}
    \item[Agilidad:] Capacidad de una organización o equipo para adaptarse rápidamente a cambios, entregar valor de forma continua y responder con flexibilidad ante nuevas necesidades.
    
    \item[Backlog:] Lista priorizada de tareas, funcionalidades o elementos de trabajo pendientes por desarrollar o ejecutar dentro de un proyecto.
    
    \item[Flujo de trabajo (Workflow):] Conjunto de etapas o pasos secuenciales que sigue una tarea desde su inicio hasta su finalización.
    
    \item[Kaizen:] Término japonés que significa “mejora continua”. En Kanban, hace referencia a la práctica constante de identificar y aplicar pequeñas mejoras para optimizar procesos.
    
    \item[Kanban:] Metodología ágil de gestión visual del trabajo basada en principios del Lean Manufacturing, que promueve la eficiencia, transparencia y mejora continua en la entrega de valor.
    
    \item[Límites WIP (Work In Progress):] Restricciones impuestas al número de tareas que pueden estar en proceso simultáneamente, con el fin de evitar sobrecarga y mejorar el flujo.
    
    \item[Lean Manufacturing:] Filosofía de gestión enfocada en la eliminación de desperdicios, mejora continua y entrega eficiente de valor al cliente.
    
    \item[Scrum:] Marco de trabajo ágil basado en iteraciones cortas llamadas sprints, con roles definidos y reuniones periódicas para planificar, revisar y ajustar el trabajo.
    
    \item[Swimlanes:] Divisiones horizontales en un tablero Kanban que agrupan tareas por tipo, responsable, prioridad u otro criterio relevante.
    
    \item[Tarjeta Kanban:] Representación visual de una tarea o unidad de trabajo. Contiene información clave como el nombre de la tarea, responsable, fecha de entrega, entre otros.
    
    \item[Visualización:] Principio fundamental de Kanban que implica representar gráficamente el trabajo para facilitar su seguimiento, priorización y toma de decisiones.
\end{description}

\vspace{0.5cm}
