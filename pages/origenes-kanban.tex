\section{Orígenes y evolución de Kanban}

La metodología Kanban, ampliamente reconocida en el ámbito de la ingeniería de software y la gestión de proyectos, tiene sus raíces en la industria manufacturera japonesa del siglo XX. Su evolución ha estado marcada por la necesidad constante de optimizar procesos, reducir el desperdicio y mejorar el flujo continuo de trabajo. A continuación, se expone su desarrollo histórico desde su concepción original hasta su adaptación a contextos modernos de trabajo del conocimiento.

\subsection{El nacimiento de Kanban en Toyota}

Kanban, palabra japonesa que significa ``señal visual'' o ``tarjeta de señalización'', fue implementada por primera vez en la planta de producción de \textbf{Toyota Motor Corporation} en la década de 1940. Este sistema surgió como respuesta al modelo de producción en masa occidental, cuya rigidez resultaba ineficiente en el contexto económico y social de la posguerra japonesa.

Taiichi Ohno, ingeniero industrial de Toyota, desarrolló el sistema Kanban como parte del \textbf{Toyota Production System (TPS)}, también conocido como producción ajustada o \textit{Lean Manufacturing}. La idea central consistía en utilizar tarjetas físicas para señalar la necesidad de reposición de insumos o productos, promoviendo así un sistema de ``pull'' (extracción) donde el trabajo o materiales solo se reponían cuando realmente eran necesarios. Esta lógica permitió reducir inventarios, eliminar cuellos de botella y fomentar una producción más ágil y eficiente.

\subsection{Principios del Lean Manufacturing que influyeron en Kanban}

El enfoque de Toyota se sustentó en dos pilares fundamentales:
\begin{itemize}
    \item \textbf{Justo a tiempo (Just-In-Time)}: producir sólo lo necesario, en el momento necesario, y en la cantidad necesaria.
    \item \textbf{Jidoka (automatización con un toque humano)}: detener la producción ante cualquier problema de calidad.
\end{itemize}

Estos principios influyeron directamente en Kanban al fomentar un control visual, la calidad desde el origen, y la mejora continua de los procesos (\textit{Kaizen}). Aunque el sistema se creó para entornos físicos, su filosofía es altamente transferible a trabajos intelectuales y digitales.

\subsection{Transición de Kanban hacia entornos de software}

Durante las primeras décadas del siglo XXI, el pensamiento Lean comenzó a permear sectores distintos al industrial. Fue \textbf{David J. Anderson}, un experto en gestión de TI, quien lideró la adaptación formal de Kanban al ámbito del desarrollo de software y la gestión de servicios.

A partir del año 2004, Anderson comenzó a aplicar los principios de Kanban en proyectos de TI, y en 2010 publicó su libro \textit{Kanban: Successful Evolutionary Change for Your Technology Business}, una obra fundamental que consolidó la metodología como herramienta ágil y evolucionaria. Anderson estableció que Kanban no requiere un cambio drástico en la estructura de trabajo existente, sino que se integra progresivamente respetando los roles, procesos y responsabilidades existentes.

\subsection{Kanban como metodología ágil}

Si bien Kanban no nació como una metodología ágil en sí, ha sido ampliamente adoptada como tal debido a que promueve los valores del \textit{Agile Manifesto}:
\begin{itemize}
    \item \textit{Colaboración por encima de contratos rígidos}.
    \item \textit{Respuesta al cambio por encima del seguimiento de un plan}.
    \item \textit{Entrega continua de valor al cliente}.
\end{itemize}

Kanban se distingue por su \textbf{enfoque incremental y evolutivo}, lo que lo diferencia de metodologías como Scrum, que suelen requerir estructuras más definidas. Esto ha facilitado su adopción en contextos donde los equipos desean mejorar sin realizar transformaciones disruptivas.

\subsection{Expansión hacia otros sectores}

Con el paso del tiempo, Kanban ha dejado de ser una herramienta exclusiva del desarrollo de software. Hoy en día, su aplicación se extiende a sectores tan diversos como:
\begin{itemize}
    \item Atención al cliente y call centers.
    \item Marketing digital.
    \item Educación y planificación académica.
    \item Gestión de operaciones y logística.
\end{itemize}

Esta expansión ha sido posible gracias a la simplicidad del sistema y a la capacidad de adaptación de sus principios a flujos de trabajo de muy distinta naturaleza.

\subsection{Resumen del proceso evolutivo}

En resumen, Kanban ha transitado por un camino evolutivo que puede sintetizarse en los siguientes hitos clave:

\begin{enumerate}
    \item \textbf{Década de 1940}: nacimiento del sistema Kanban en Toyota como parte del TPS.
    \item \textbf{Década de 1980-1990}: difusión del pensamiento Lean a nivel mundial.
    \item \textbf{Década de 2000}: transición del enfoque hacia la industria del software.
    \item \textbf{2010 en adelante}: consolidación como metodología ágil y expansión hacia múltiples industrias.
\end{enumerate}

El conocimiento de este proceso es indispensable para comprender la filosofía que sustenta Kanban, así como sus potencialidades y límites en distintos entornos laborales.

