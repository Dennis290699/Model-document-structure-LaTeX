\section{Comparación con otras metodologías ágiles}

El ecosistema de metodologías ágiles ofrece diversos enfoques para la gestión eficiente de proyectos, cada uno con características particulares que los hacen más o menos adecuados según el tipo de equipo, producto o contexto organizacional. En esta sección se analiza comparativamente a Kanban frente a otras metodologías ágiles ampliamente utilizadas, tales como Scrum y eXtreme Programming (XP), resaltando similitudes, diferencias y ámbitos de aplicación.

\subsection{Kanban vs Scrum}

\textbf{Scrum} es probablemente la metodología ágil más difundida. Se estructura en ciclos temporales fijos denominados \textit{sprints}, que suelen tener una duración de entre 1 y 4 semanas. Incluye roles definidos (Scrum Master, Product Owner, equipo de desarrollo) y una serie de eventos obligatorios (Daily Scrum, Sprint Planning, Sprint Review, Sprint Retrospective).

Por el contrario, \textbf{Kanban} es menos prescriptivo y no impone iteraciones fijas ni roles formales. Se centra en el flujo continuo del trabajo y en la mejora progresiva basada en la visualización y medición del sistema actual.

\begin{itemize}
    \item \textbf{Estructura temporal:} Scrum trabaja en bloques temporales cerrados; Kanban es continuo y flexible.
    \item \textbf{Roles:} Scrum define roles estrictos; Kanban los deja abiertos según el contexto.
    \item \textbf{Planeación:} Scrum realiza una planificación exhaustiva al inicio del sprint; Kanban permite añadir y priorizar tareas de forma dinámica.
    \item \textbf{Adaptabilidad:} Kanban es más adecuado cuando los requisitos cambian constantemente; Scrum funciona mejor con requisitos más estables por iteración.
\end{itemize}

\subsection{Kanban vs eXtreme Programming (XP)}

\textbf{XP (eXtreme Programming)} es una metodología enfocada en mejorar la calidad del software mediante prácticas técnicas rigurosas como programación en pareja, desarrollo guiado por pruebas (TDD) y entrega continua. A diferencia de Kanban, XP está fuertemente ligado al ámbito técnico y al desarrollo de software como actividad central.

\begin{itemize}
    \item \textbf{Enfoque técnico:} XP es intensamente técnico; Kanban es más general y aplicable a múltiples dominios.
    \item \textbf{Prácticas:} XP prescribe un conjunto de buenas prácticas específicas; Kanban ofrece principios adaptables al contexto.
    \item \textbf{Flexibilidad:} Kanban permite introducir mejoras de forma progresiva sin grandes cambios iniciales; XP requiere el compromiso con sus prácticas desde el inicio.
\end{itemize}

\subsection{¿Cuándo usar Kanban?}

Kanban es especialmente recomendable en los siguientes escenarios:

\begin{itemize}
    \item Cuando los requerimientos cambian frecuentemente y no es posible planificar con anticipación mediante iteraciones cerradas.
    \item En equipos que ya cuentan con un proceso de trabajo, pero desean introducir mejoras progresivas sin rupturas.
    \item En proyectos de soporte, mantenimiento, atención al cliente u operaciones, donde el flujo continuo es prioritario.
    \item Cuando se busca una transición hacia un modelo ágil sin una reestructuración profunda del equipo o proceso actual.
\end{itemize}

\subsection{Síntesis comparativa}

\begin{longtable}{|p{4cm}|p{4.5cm}|p{4.5cm}|}
\hline
\textbf{Criterio} & \textbf{Scrum} & \textbf{Kanban} \\
\hline
Estructura & Iteraciones fijas (sprints) & Flujo continuo \\
\hline
Roles & Definidos y obligatorios & No prescriptivo \\
\hline
Planificación & Por sprint & Adaptativa y continua \\
\hline
Cambio de tareas & Limitado por sprint & Permitido en cualquier momento \\
\hline
Aptitud para cambios frecuentes & Media & Alta \\
\hline
Curva de aprendizaje & Alta & Baja \\
\hline
Ámbito técnico & Moderado & Amplio y general \\
\hline
\end{longtable}

