\documentclass[a4paper,12pt]{article}
\usepackage[utf8]{inputenc}
\usepackage[T1]{fontenc}
\usepackage[spanish]{babel}
\usepackage{hyperref}
\usepackage{geometry}
\usepackage{fancyhdr}
\usepackage{titlesec}
\usepackage{graphicx}
\usepackage{amsmath}
\usepackage{listings}
\usepackage{longtable}
\usepackage{float}

% Configuración de márgenes
\geometry{top=2.5cm, bottom=2.5cm, left=3cm, right=3cm}

% Configuración de encabezados y pie de página
\pagestyle{fancy}
\fancyhf{}
\fancyhead[L]{\textit{Titulo del proyecto}}
\fancyhead[R]{\thepage}
\fancyfoot[C]{\textit{Nombre de la catedra}}

% Formato de títulos
\titleformat{\section}{\large\bfseries}{\thesection.}{0.5em}{}
\titleformat{\subsection}{\normalsize\bfseries}{\thesubsection.}{0.5em}{}

% Configuración del índice
\usepackage{tocloft}
\renewcommand{\cftsecfont}{\bfseries}
\renewcommand{\cftsubsecfont}{\itshape}
\renewcommand{\cfttoctitlefont}{\Large\bfseries}

% Configuración de listings
\usepackage{xcolor}

\lstset{
    inputencoding=utf8,
    extendedchars=true,
    literate={├}{{\textbar}}1
             {─}{{\textendash}}1
             {│}{{\textbar}}1
             {└}{{\textbar}}1,
    basicstyle=\ttfamily\footnotesize, % Fuente monoespaciada y tamaño pequeño
    keywordstyle=\bfseries\color{blue}, % Palabras clave en negrita y azul
    commentstyle=\itshape\color{green!50!black}, % Comentarios en cursiva y verde oscuro
    stringstyle=\color{red}, % Cadenas en rojo
    numberstyle=\tiny\color{gray}, % Números de línea en gris y tamaño pequeño
    backgroundcolor=\color{gray!10}, % Fondo gris claro
    frame=shadowbox, % Marco sombreado
    rulesepcolor=\color{gray!20}, % Color del borde interno del marco
    breaklines=true, % Permitir líneas largas divididas
    numbers=left, % Numeración de líneas a la izquierda
    captionpos=b, % Posición del título del código (abajo)
    showspaces=false, % No mostrar espacios como símbolos
    showstringspaces=false, % No mostrar espacios en cadenas
    tabsize=4, % Tamaño del tabulador
    morekeywords={self, with, as}, % Agregar más palabras clave si es necesario
}

\begin{document}

% Portada
\begin{titlepage}
    \centering
    \vspace*{1.5cm}

    % Título del proyecto (MODIFICAR)
    {\LARGE\bfseries [Título del Proyecto] \par}
    \vspace{0.5cm}
    {\large [Subtítulo del Proyecto, si aplica] \par}

    \vspace{0.8cm}
    \includegraphics[width=0.3\textwidth]{[Ruta del Logo de la Institución]}
    \vspace{0.5cm}

    \vfill

    % Integrantes (MODIFICAR)
    % Integrantes
    {\normalsize \textbf{Integrantes:} \par}
    \vspace{0.3cm}
    \begin{minipage}{0.7\textwidth}
    \centering
    \begin{tabular}{ll}
        Apellidos completos & Nombres Completos \\
        Apellidos completos & Nombres Completos \\
        Apellidos completos & Nombres Completos \\
        Apellidos completos & Nombres Completos \\
        Apellidos completos & Nombres Completos \\
        Apellidos completos & Nombres Completos \\
        Apellidos completos & Nombres Completos \\
        Apellidos completos & Nombres Completos \\
        Apellidos completos & Nombres Completos \\
        Apellidos completos & Nombres Completos \\
    \end{tabular}
    \end{minipage}

    \vspace{1cm}
    {\normalsize \textbf{Fecha:} \today \par}
    \vspace{1cm}

    \vfill

    % Pie de página institucional (MODIFICAR si es necesario)
    {\small\textit{[Nombre de la Universidad] \\ [Nombre de la Facultad o Departamento] \\ [Nombre de la Cátedra o Asignatura]}}
\end{titlepage}

% Abstract
\section*{Resumen}

Este informe académico presenta un estudio detallado y profundo sobre la metodología ágil \textbf{Kanban}, abordando su origen, evolución, fundamentos teóricos, componentes clave, fases del flujo de trabajo, así como su implementación práctica en diversos contextos. La investigación ha sido desarrollada de manera colaborativa por un equipo de estudiantes de Ingeniería de Software, con el objetivo de comprender y analizar el impacto de Kanban en la gestión eficiente de proyectos, particularmente en el desarrollo de software.

Kanban, cuyo origen se remonta al sistema de producción de Toyota bajo los principios del \textit{Lean Manufacturing}, ha sido adaptado exitosamente al ámbito de la ingeniería del software y la gestión de proyectos. Su enfoque basado en la visualización del trabajo, la limitación del trabajo en curso (WIP), la gestión continua del flujo y la mejora constante (Kaizen), ha permitido a las organizaciones alcanzar mayores niveles de eficiencia, flexibilidad y transparencia.

A lo largo del documento se exploran los principios fundamentales de Kanban, sus elementos estructurales (como los tableros, tarjetas, columnas y límites WIP), las fases típicas del flujo de trabajo y las buenas prácticas para su implementación. Además, se realiza una comparación analítica con otras metodologías ágiles como Scrum y XP, destacando las ventajas relativas de Kanban, especialmente en entornos donde se requiere adaptabilidad progresiva sin cambios disruptivos.

Complementariamente, se incluyen casos de estudio y ejemplos reales que ilustran cómo Kanban ha sido aplicado con éxito tanto en el sector tecnológico como en otros ámbitos organizacionales. Finalmente, se reflexiona sobre las lecciones aprendidas y la relevancia de Kanban como herramienta de transformación organizacional y como competencia esencial para los profesionales de la ingeniería de software del siglo XXI.

\vspace{0.5cm}
\noindent\textbf{Palabras clave:} Kanban, metodologías ágiles, gestión de proyectos, flujo de trabajo, mejora continua, Lean, software.


% Índice
\tableofcontents
\newpage

% Secciones del documento
\section{Introducción}

En el actual panorama de la ingeniería de software, donde la adaptabilidad, la eficiencia y la entrega continua de valor son requisitos esenciales, las metodologías ágiles han emergido como enfoques dominantes para la gestión de proyectos. Dentro de este ecosistema ágil, \textbf{Kanban} destaca por su enfoque visual, flexible y progresivo hacia la optimización del flujo de trabajo.

La presente investigación ha sido desarrollada de manera colaborativa por un grupo de estudiantes con el objetivo de ahondar en la comprensión estructural, operativa y práctica de Kanban. Esta metodología, originada en el sistema de producción de Toyota, ha trascendido su contexto industrial para convertirse en una herramienta de gran utilidad en el desarrollo de software, la atención al cliente, el marketing, la educación, entre otros sectores. Su capacidad para adaptarse sin necesidad de cambiar completamente la estructura existente de trabajo la hace especialmente atractiva para equipos que buscan mejorar sin interrupciones bruscas.

El informe que a continuación se presenta responde a la necesidad académica de adquirir conocimientos robustos sobre los marcos ágiles contemporáneos, en específico, sobre Kanban. A lo largo del documento, se exploran sus antecedentes, principios fundamentales, componentes clave, fases del flujo de trabajo, roles posibles, herramientas digitales de apoyo, ventajas comparativas, y se ilustran sus beneficios mediante ejemplos reales de implementación. 

Además, este trabajo no solo tiene un valor teórico, sino también formativo. Cada sección ha sido elaborada con una intención pedagógica clara, motivando el trabajo colaborativo, la investigación individual y el análisis crítico por parte del grupo, en concordancia con los objetivos de la cátedra de Ingeniería de Software.

Así, este estudio se convierte en una herramienta de doble propósito: aportar al conocimiento colectivo del grupo investigador y ofrecer a otros estudiantes una guía clara y detallada sobre Kanban y su aplicabilidad práctica.




\section{Conclusiones}

El presente informe ha permitido realizar un análisis riguroso, detallado y multidimensional de la metodología ágil Kanban, desde sus orígenes industriales hasta su consolidación como herramienta estratégica en la gestión de proyectos y equipos de trabajo. A través de la investigación y el estudio de casos reales, se ha demostrado que Kanban no solo es una técnica visual de organización, sino una filosofía de mejora continua orientada a la eficiencia operativa y la transparencia colaborativa.

En primer lugar, la comprensión de los principios fundamentales de Kanban —como la visualización del flujo de trabajo, la limitación del trabajo en curso (WIP) y el fomento de ciclos de retroalimentación— permite al equipo adoptar prácticas de gestión adaptativas y menos prescriptivas que otras metodologías ágiles. Este enfoque, al centrarse en la evolución progresiva y no en revoluciones estructurales, facilita su adopción incluso en organizaciones con escasa experiencia previa en agilidad.

Asimismo, el desglose del flujo de trabajo en fases claras, acompañado del uso de tableros físicos o digitales, ha demostrado su eficacia en sectores tan diversos como el desarrollo de software, la atención al cliente y la gestión de campañas publicitarias. Esta adaptabilidad subraya la naturaleza versátil de Kanban, la cual lo hace apto tanto para entornos dinámicos como para contextos donde se requiere una estructura más estable.

Otro aspecto destacable es la flexibilidad de roles y la ausencia de jerarquías rígidas, lo cual empodera a los miembros del equipo, promueve la autorregulación y fortalece la colaboración horizontal. No obstante, la correcta implementación de Kanban exige un compromiso organizacional con la transparencia, el análisis continuo de cuellos de botella y la disposición para ajustar procesos sobre la marcha.

En términos comparativos, Kanban se diferencia de metodologías como Scrum por su enfoque evolutivo, lo cual lo convierte en una alternativa viable para equipos que requieren un modelo de transición más suave hacia la agilidad.

Finalmente, esta investigación grupal nos ha permitido consolidar un aprendizaje integral no solo sobre la técnica en sí, sino también sobre el valor de trabajar colaborativamente, analizar fuentes académicas de calidad y plasmar conocimientos de forma estructurada y crítica. En un entorno académico y profesional cada vez más orientado a la adaptabilidad y la eficiencia, dominar herramientas como Kanban representa una competencia clave para los futuros ingenieros de software.

\section{Bibliografía}

\begin{itemize}
    \item Ahmad, M. O., Markkula, J., \& Oivo, M. (2021). Kanban in software engineering: A systematic mapping study. \textit{Journal of Systems and Software}, 181, 111028. https://doi.org/10.1016/j.jss.2021.111028

    \item Corona, E., \& Pani, F. E. (2020). A review of lean–kanban approaches in software development. \textit{Journal of Software: Evolution and Process}, 32(1), e2207. https://doi.org/10.1002/smr.2207

    \item González, P., \& Jiménez, S. (2023). Aplicación de metodologías ágiles: Un enfoque práctico de Kanban en proyectos tecnológicos. \textit{Revista Latinoamericana de Ingeniería de Software}, 5(2), 45-59.

    \item Sutherland, J., \& Schwaber, K. (2020). \textit{The Scrum Guide}. Scrum.org. Recuperado de \url{https://scrumguides.org/}

    \item Martínez, A., \& López, R. (2022). Kanban como estrategia para la mejora del flujo de trabajo en organizaciones ágiles. \textit{Revista Iberoamericana de Gestión de Proyectos}, 11(1), 33-48. https://doi.org/10.31039/ijocs.2022.11.1.3

    \item Project Management Institute. (2021). \textit{Guía de los Fundamentos para la Dirección de Proyectos (Guía del PMBOK)} (7ª ed.). Project Management Institute.

    \item Kniberg, H. (2022). \textit{Kanban and Scrum: How to Make the Most of Both} (2ª ed.). C4Media.
\end{itemize}


\input{pages/anexos}

\end{document}
